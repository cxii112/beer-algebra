\section*{Примечания}
Первое определение множества было иным.
\begin{align*}
  l \in \mathbb{R} - \text{объем пива в литрах,} \\
  d \in [0, +\infty)- \text{масса алкоголя.}
\end{align*}
\begin{align*}
   & \forall x, y \in B:                                             \\
   & \begin{cases}
       l_x \neq 0 \lor l_y \neq 0 & (l_x, d_x) + (l_y, d_y) =
       \left(l_x + l_y, \frac{|l_x|d_x + |l_y|d_y}{|l_x + l_y|}\right) \\
       l_x = 0,l_y = 0            & (0, d_x) + (0, d_y) =
       \left(0, 0\right)                                               \\
       l_x = l, l_y = -l          & (l, d_x) + (-l, d_y) =
       \left(0, 0\right)
     \end{cases}
\end{align*}
Ну и умножение на скаляр:
\begin{align*}
  (\forall x,y \in B)(\forall \alpha \in \mathbb{R}):
  \alpha(l_x, d_x) = (\alpha l_x, d_x) \\
\end{align*}
Причем это уже не первый вариант, в самом первом $d \in (0,+\infty)$ чтобы уже по определению исключить безалкогольное пиво, но ничего не вышло. Короче, менял я и определение множества и определения операций, чаще всего множество было линейным пространством, но вот не было евклидовым. В итоге я решил взять те определения, что сейчас.

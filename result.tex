\section*{Результат}
Используя полученное скалярное произведение найдем \textit{норму пива}.
\begin{align*}
  \forall b \in B: \|b\| = |\sqrt{b^{T}\Gamma b}|
\end{align*}
А теперь найдем норму \textbf{Балтики 9}.
\begin{align*}
  \left(\frac12,\frac9{200}\right) - \text{наша Балтика 9} \\
  \begin{pmatrix}\frac12 & \frac9{200} \end{pmatrix}
  \mtx
  \begin{pmatrix}\frac12 \\ \frac9{200} \end{pmatrix} =    \\
  =\begin{pmatrix}\frac{\mtxa}2 & \frac9{200}\cdot\mtxb \end{pmatrix}
  \begin{pmatrix}\frac12 \\ \frac9{200} \end{pmatrix} =    \\
  = \frac74 + 50\cdot\frac9{200} = \frac74 + \frac94 = 4
\end{align*}
\begin{align*}
  \left\|\left(\frac12,\frac9{200}\right)\right\|^2 = 4 \\
  \left\|\left(\frac12,\frac9{200}\right)\right\| = |\sqrt4| = 2
\end{align*}
\textbf{Таким образом норма Балтики 9 - 2.}

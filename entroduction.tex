\section*{Вступление}
Найти норму пива довольно просто. Опишу кратко алгоритм действий.

Зададим пиво как вектор и определим операции над ним.
Из векторов молучим множество, с которым и будем работать.

Докажем, что множество - линейное пространство.
Определим операцию скалярного произведения и проверим,
что она работает для всех элементов пространства.

Таким образом мы получим евклидово пространство пива.
В евклидовом пространтве можно найти норму вектора.
А наш вектор - пиво. Так и будем искать \textit{норму пива}.
